\chapter{Réalisations}
%Si besoin, vous pourriez structurer le reste du rapport en plusieurs parties et non une seule. La ou les parties devraient elles-mêmes êtres structurées en plusieurs sous-sections au sein d'une même partie. Dans tous les cas, la logique du plan doit apparaître clairement.

%Travaillez les liaisons pour aboutir à une lecture fluide. Voici un exemple (un peu exagéré) : "Après avoir inventorié les technologies disponibles dans la section précédente, cette section est consacrée aux expérimentations que nous avons menées avec chacune d'elles. Ce travail nous permettra de sélectionner les technologies retenues, présentées dans la section suivante."

%Présentez votre réflexion et vos choix, qui devraient être justifiés. Examinez rapidement les autres alternatives.

%Sélectionnez les détails pertinents et laissez les autres en annexe. Allez du général au particulier. Evitez de présenter un catalogue des fonctions développées.
Le sujet de mon stage étant large, très tôt j'ai dû me concentrer sur plusieurs tâches puis rapidement m'atteler à plusieurs réalisations afin d'assurer au mieux ma fonction. Nous verrons donc dans cette partie, pour chacune de ces différentes réalisations, les besoins auxquels il fallait répondre, les choix qui ont été fait et les différentes difficultés rencontrées.

\section{Conception d'un processus de production et de gestion des rapports de test}
%QQchose ?
\subsection{État des lieux}
Dans un premier temps, une présentation du contexte initial est indispensable. Au sein de l'équipe \gls{TC}, il existe différentes phases de test, dont l'organisation correspond au cycle en V du projet (cf. Annexe I) :
\begin{itemize}
\item \textbf{FAT} : Factory Acceptance Test
\item \textbf{iFAT} : Integrated Factory Acceptance Test
\item \textbf{\gls{StAT}} : Stand Alone Test
\item \textbf{\gls{SIT}} : Site Integration Test 
\item \textbf{\gls{E2E}} : End to End Test 
\end{itemize}

Nous nous concentrons ici sur les phases \gls{StAT}, \gls{SIT}, \gls{E2E}. 

Pour chaque sous-système du projet, 3 procédures de test ont été rédigées, une par phase de test. 

Ainsi, si une localisation du projet abrite 3 sous-systèmes nécessitant d'être testés, il y aura 9 tests à effectuer dans cette localisation.

Ensuite, chaque test doit être formalisé par un rapport de test constitué de la manière suivante : 

\begin{itemize}
\item Une première page que l'on appelle ITR, qui permet d'identifier le rapport (numéro unique), d'identifier le test (localisation, phase, système, date), identifier le statut (accepté avec, sans réserves ou refusé), résumer les observations et réserves des 3 parties et consigner leurs signatures.
\item Les pages suivantes sont les \gls{TestCases}, en français les scénarios de test, qui permettent de simuler plusieurs situations d'utilisation des systèmes de manière à prouver leur conformité vis à vis de toutes les exigences du cahier des charges.
\begin{itemize}
\item Chaque \gls{TestCases} est composé de une ou plusieurs étapes qui elles mêmes sont reliées à une ou plusieurs exigences.
\end{itemize}
\end{itemize}

Parmi ces exigences, certaines sont liées à la sécurité, dans le cadre des \gls{RAMS}, et sont indexées dans un document appelé le \gls{SC}. Ces exigences liées à la sécurité sont séparés en deux \gls{SIL}, ou niveaux de criticité : SIL0, le plus faible et SIL2 le plus élevé.

Une fois ce contexte clarifié, nous avons pu entreprendre l'élaboration du processus et c'est le sujet de la prochaine section.

\subsection{Première version}
de la nécéssité de Mezzoteam et eTOL

Les exigences liées à la sécurité nécessitent d'être suivie de manière prioritaire car le futur opérateur refusera de commencer la période de formation à l'exploitation du système de ses employés tant qu'il n'aura pas la preuve de la conformité du projet vis à vis des ces exigences.

Scopes
Workflow 1
\subsection{Évolution du processus}

Validation des requirements.
IO point et documentation externe 
Snags
Workflow 2

\section{Mise au point d'un outil de collecte de données }

\gls{scraping}

\subsection{Les motivations}
1500 rapports
Plus de 15 000 TC (formulaire)
Plus de 60 000 steps (étapes)
Après avoir importé manuellement les données des premiers rapports, .. tourné vers SnagR :
Présentation des activités
Problème de cohérence
=> KPI Fausse
Absence de validation des données
Exportation des données limitées
Donc besoin d'implémenter moi même
\subsection{Les différentes fonctionnalités de l'outil}
Sauvegarde complète
Actualisation en deux étapes
\subsection{Les choix techniques}
DB, Access, powerbi ou excel ?
=> Limitations dues au caractère sensible des données, process beaucoup trop long

Pandas

Multi thread ou multi process ?
=> 
Temps d'éxécution : 1h30 => 5 min (laisser qq latences)

Python 3 ou le choix de la pérénité
=> 2to3

\section{Communication sur l'avancement du projet}
\subsection{Organisation des données}
Séparation
Séléction
\subsection{Communication interne}
Outils d'aide à la décision
KPI
S-Curves
Syncronisation avec le Testing pour plus de cohésion
\subsection{Communication avec le client}
Difficultés de la position de Thales


\section{Formation et transmission}
\subsection{Sensibilisation des membre de l'équipe }
Sensibilisation à l'importance des rapports
Formation à la validation des données (doc, convention de nommage)
\subsection{Tuilage}
Doc
Etat de l'avencement
Vidéos tutorielles


