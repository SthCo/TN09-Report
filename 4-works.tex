\chapter{Réalisations}
%Si besoin, vous pourriez structurer le reste du rapport en plusieurs parties et non une seule. La ou les parties devraient elles-mêmes êtres structurées en plusieurs sous-sections au sein d'une même partie. Dans tous les cas, la logique du plan doit apparaître clairement.

%Travaillez les liaisons pour aboutir à une lecture fluide. Voici un exemple (un peu exagéré) : "Après avoir inventorié les technologies disponibles dans la section précédente, cette section est consacrée aux expérimentations que nous avons menées avec chacune d'elles. Ce travail nous permettra de sélectionner les technologies retenues, présentées dans la section suivante."

%Présentez votre réflexion et vos choix, qui devraient être justifiés. Examinez rapidement les autres alternatives.

%Sélectionnez les détails pertinents et laissez les autres en annexe. Allez du général au particulier. Evitez de présenter un catalogue des fonctions développées.

\section{Un stage porté à la fois sur la technique}

\subsection{L'initialisation du processus}

DB, Access, powerbi ou excel ?
=> Limitations dues au caractère sensible des données, process beaucoup trop long

\subsection{Son automatisation}
Pandas

\subsection{Sa transmission}
Multi thread ou multi process ?
=> 

Python 3 ou le choix de la pérénité
=> 2to3

\section{Sur la gestion de projet}

\subsection{Les interactions avec les différentes instances Thales}

\subsection{Celles avec le client}