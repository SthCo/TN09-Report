\chapter*{Conclusion}

\addcontentsline{toc}{chapter}{Conclusion}

%En général, on commence par présenter un résumé du rapport puis les perspectives et éventuellement les travaux restant à mener.
%Vous pouvez ensuite exposer les points positifs et négatifs de votre stage.
%Enfin, vous pouvez re-situer votre stage dans votre parcours de formation et dans votre projet professionnel. Vos objectifs ont-ils évolué ? Par exemple, en quoi ce stage confirme (ou infirme) votre choix de filière ?

Ici j'exposerais :
\begin{itemize}
\item un résumé du rapport
\item une liste de travaux que j'aurai pus réaliser si j'étais resté plus longtemps (par exemple proposer de mettre l'outil en production en local sur le réseau Thales)
\item Les aspects positifs (expérience, communication, etc) et ceux négatifs(gestion de mes tâches pas optimales, mauvaise gestion de mon temps ce qui me faisait travailler le week-end ou tard le soir) de mon stage. Je vais reprendre ce que tu (Vincent) avais mis dans ma fiche parce que ça résume bien les difficultés que j'ai eues.
\item Expliquer en quoi ce stage s'est inscrit dans mon parcours : réutilisation des notions et pratiques étudiées en cours et projet (base de donnée, optimisation, gestion de projet, etc..)
\item En quoi ce stage a fait évoluer mon projet professionnel donc par exemple : il a renforcé ma motivation à travailler dans l'informatique des réseaux, celle de m'expatrier (volonté d'effectuer un VIE), ma motivation à travailler pour Thales.
\end{itemize}
