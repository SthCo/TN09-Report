\chapter*{Conclusion}

\addcontentsline{toc}{chapter}{Conclusion}

%En général, on commence par présenter un résumé du rapport puis les perspectives et éventuellement les travaux restant à mener.
%Vous pouvez ensuite exposer les points positifs et négatifs de votre stage.
%Enfin, vous pouvez re-situer votre stage dans votre parcours de formation et dans votre projet professionnel. Vos objectifs ont-ils évolué ? Par exemple, en quoi ce stage confirme (ou infirme) votre choix de filière ?

% un résumé du rapport
Ce stage a été l'occasion de découvrir un projet aussi complexe que passionnant, tant par les différents environnements dans lesquelles j'ai évolué (bureaux administratifs, bureaux opérationnels, centre logistique, site) que par les responsabilités que l'on m’a confiées (suivre la production des rapports, aller à l'encontre du client, prendre en charge le bon fonctionnement de la base de données SnagR, former un autre employé).
% une liste de travaux que j'aurai pus réaliser si j'étais resté plus longtemps (par exemple proposer de mettre l'outil en production en local sur le réseau Thales)
De cette richesse découle un constat : j'aurais pu améliorer mes réalisations, implémenter de nouvelles fonctions pour l'outil d'extraction des données que j'ai conçu.

\begin{itemize}
\item J'aurais pu utiliser le réseau local du projet pour exécuter l'extraction des données en tâches de fond ainsi que des sauvegardes journalières. Cela aurait permis de rendre accessibles à tout moment les données à jour ainsi que la dernière sauvegarde des rapports.
\item Puis, à l'aide de macros Excel, j'aurai eu l'opportunité de permettre à la base de données et donc aux indicateurs que j'ai mis au point de s'actualiser automatiquement.
\item En ce qui concerne le problème de l'absence de validation des données saisies par les utilisateurs, j'aurais pu trouver une fonction permettant d'importer les données en corrigeant les fautes de frappe ou les erreurs de formats.
\end{itemize}

% Les aspects positifs (expérience, communication, etc) et ceux négatifs(gestion de mes tâches pas optimales, mauvaise gestion de mon temps ce qui me faisait travailler le week-end ou tard le soir) de mon stage. Je vais reprendre ce que tu (Vincent) avais mis dans ma fiche parce que ça résume bien les difficultés que j'ai eues.
En ce qui concerne mon analyse personnelle de ce stage, je pense qu'il a été l'occasion de développer ou de renforcer des compétences indispensables à la vie professionnelle d'un ingénieur, dont en voici quelques-unes :
\begin{itemize}
\item \underline{La communication en entreprise :} J'ai pu me rendre compte du caractère indispensable de l'esprit de synthèse que ce soit dans le cadre d'un mail, d'une question à un collègue ou bien d'une intervention en réunion.
\item \underline{S'exprimer en anglais :} bien que l'expression écrite ne diffère pas énormément entre celle en cours et celle en entreprise, c'est l'expression orale dans un projet à l'étranger qui marque une vraie différence. En effet, les différents accents et champs lexicaux rencontrés diffèrent en fonction des origines de chacun (Philippins, Anglos-Saxons, Indiens, Égyptiens, Algériens, etc.). Il faut pouvoir s'y habituer rapidement afin d'assurer un dialogue intelligible pour les deux parties.
\item \underline{L'importance de l'autonomie :} même si mon maître de stage ainsi que toute l'équipe se sont rendus disponibles pendant l'intégralité de mon stage, il y a eu des situations où d'autres acteurs ne répondaient pas à mes questions et où j'ai bien fait, il me semble, de décider de trouver des réponses par moi même (par exemple lorsque j'ai cherché à obtenir une documentation pour l'API version 2,  c.f 3.2.1).
\end{itemize}

De même, j'ai pu identifier des points où je gagnerais à m'améliorer :
\begin{itemize}
\item \underline{L'organisation du temps de travail :} et plus particulièrement l'ordonnancement des tâches. Effectivement, je m'attelais parfois à des tâches peu prioritaires, car j'avais toutes les ressources nécessaires pour les réaliser alors que d'autres tâches bien plus prioritaires devaient être livrées le lendemain (par exemple le rapport d'avancement du reporting). Ce qui me menait parfois à veiller tard dans la nuit pour réaliser la tâche à livrer en priorité.
\item \underline{Garder son sang-froid :} dans certaines réunions avec le client, je prenais trop à cœur certaines discussions et positions tenues par ce dernier (sans pour autant commettre d'impair). Pourtant, dans un projet de cette envergure, il est important d'arriver à dissocier le travail et les émotions.
\end{itemize}

% Expliquer en quoi ce stage s'est inscrit dans mon parcours : réutilisation des notions et pratiques étudiées en cours et projet (base de donnée, optimisation, gestion de projet, etc..)
% En quoi ce stage a fait évoluer mon projet professionnel donc par exemple : il a renforcé ma motivation à travailler dans l'informatique des réseaux, celle de m'expatrier (volonté d'effectuer un VIE), ma motivation à travailler pour Thales.

Pour conclure, je pense que ce stage s'est inscrit parfaitement à la fois dans mon parcours et mon projet professionnel.
Plus précisément, j'ai réutilisé des connaissances et pratiques acquises en cours (notions de base de données avec NF17, de gestion de projet avec GE37, d'algorithmique avec NF16 et d'optimisation avec RO03).

De plus, ce stage a renforcé ma volonté d'évoluer au sein de la filière SRI bien que ce ne soit pas le domaine dans lequel j'ai travaillé : c'était celui de nombre de mes collègues qui n'ont cessé de piquer ma curiosité concernant leur travail sur le réseau et la cybersécurité.

Enfin, ma volonté d'expatriation ressort grandie, je nourris même le vœu de réaliser un VIE auprès du groupe Thales après avoir obtenu mon diplôme.
