\chapter*{Résumé technique}
\addcontentsline{toc}{chapter}{Résumé technique}

Lors de mon TN09, j'ai effectué mon stage chez Thales Gulf, à Doha, au Qatar. Le projet auquel j'ai été affecté est un projet ferroviaire (\gls{LRT}) pour lequel Thales prend en charge la majorité des technologies de télécommunications. L'équipe \gls{TC}, dans laquelle j'ai évolué, est chargée de mettre en service et de tester les différents équipements des nombreux systèmes gérés par Thales. \\
Ma fonction principale était de rapporter l'avancement de la production des rapports de tests, afin, par exemple, de fournir des outils d'aide à la décision, ainsi que des indicateurs de progrès, à mes managers \\
Afin de mener à bien cette mission, j'ai utilisé 2 bases de données différentes. L'une regroupait tous les documents produits au cours du projet, l'autre permettait aux différentes équipes d'ingénieurs d'éditer et de stocker leurs rapports de tests. \\
Cependant, c'est sur la seconde base de données que la majorité de mon travail s'est concentrée. En effet, les informations indexées dans celle-ci correspondaient à une extraction des données brutes présentes dans les formulaires composant chaque rapport. Ainsi, un de mes projets fut de mettre au point un programme capable d'analyser ces données, les relier aux différentes variables , puis les intégrer au sein d'un fichier Excel afin de pouvoir fournir différentes statistiques et outils d'aide à la décision.\\
Lors de ce stage, j'ai pu découvrir de nombreuses structures d'entreprise ainsi que des domaines d'ingénierie informatiques divers et variés, mais aussi, comme vous pourrez le constater dans la suite de ce rapport, j'ai endossé de nombreuses responsabilités, qui ont contribué à rendre cette expérience formatrice. \\
Dans ce rapport, je m'efforcerai de vous présenter le contexte de mon stage, la nature de mes missions, mes réalisations et enfin mon ressenti ainsi que l'expérience que j'en ai tirée.
