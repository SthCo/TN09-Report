\usepackage[toc,section=chapter]{glossaries}    % Glossaire

\makeglossaries

\newglossaryentry{TC}
{
	name=\underline{T\&C},
	description={\textit{Testing \& Commissioning}, département d'un projet dédié à l'inspection et à la mise en service des divers équipements et systèmes du projet}
}

\newglossaryentry{StAT}
{
	name=\underline{StAT},
	description={\textit{Stand Alone Test}, test unitaire, où l'équipement testé est isolé de toute interaction durant le test, afin de restreindre les critères du test à son fonctionnement autonome  }
}

\newglossaryentry{LRT}
{
	name=\underline{LRT},
	description={\textit{Light Rail Transit}, forme de transport en commun urbain ferroviaire disposant généralement d'une capacité et d'une vitesse inférieures à celles d'un train ou d'un métro, mais supérieures à celles des systèmes traditionnels de tramway. (c.f  \underline{\href{https://fr.wikipedia.org/wiki/Métro_léger}{Wikipédia}})}
	}

\newglossaryentry{TCS}
{
	name=\underline{TCS},
	description={\textit{Tramway Control System}, Système de Contrôle du Tramway}
}

\newglossaryentry{RST}
{
	name=\underline{RST},
	description={\textit{Rolling Stock}, ensemble du matériel roulant, ici c'est l'ensemble des différentes rames du LRT}
}

\newglossaryentry{PSD}
{
	name=\underline{PSD},
	description={\textit{Platform Screen Door}, portes palières ou encore façades de quai sont des portes automatiques vitrées situées le long des quais en bordure des voies ne s'ouvrant que lorsque la rame est à l'arrêt en station (c.f  \underline{\href{https://fr.wikipedia.org/wiki/Porte_palière_(métro)}{Wikipédia}})}
}

\newglossaryentry{ECS}
{
	name=\underline{ECS},
	description={\textit{Environmental Control System}, Système de Contrôle Environnemental}
}

\newglossaryentry{TVS}
{
	name=\underline{TVS},
	description={\textit{Tunnel Ventilation System},  Système de Ventilation des Tunnels }
}

\newglossaryentry{CCS}
{
	name=\underline{CCS},
	description={\textit{Communication and Control System}, Système de Contrôle et de Communication, département du projet concernant principalement les télécommunications, la sécurité et le monitoring}
}

\newglossaryentry{TETRA}
{
	name=\underline{TETRA},
	description={\textit{Terrestrial Trunked Radio}, système de radio numérique mobile professionnel bi-directionnel, spécialement conçu pour des services officiels et pour l'armée. Un réseau de type TETRA offre un canal radio partagé ouvert en permanence, et réservé à un groupe d'utilisateurs. Ceci permet d'établir une communication immédiate entre un utilisateur sur le terrain et un dispatcher, ou un groupe d'utilisateurs (c.f \underline{\href{https://fr.wikipedia.org/wiki/Terrestrial_Trunked_Radio}{Wikipédia}})}
}

\newglossaryentry{DTS}
{
	name=\underline{DTS},
	description={\textit{Digital Transmission System}, Système de Transmission Digitale, ensemble des différentes infrastructures réseau}
}

\newglossaryentry{COMTV}
{
	name=\underline{COMTV},
	description={\textit{Commercial Television}, Télévision commerciale, désigne l'ensemble des écrans situés en station et à bord des rames ayant pour fonction de diffuser des annonces publicitaires aux usagers}
}

\newglossaryentry{UPS}
{
	name=\underline{UPS},
	description={\textit{Uninterruptible Power Supply}, L'Alimentation Sans Interruption (ASI), ou encore un onduleur, est un dispositif de l'électronique de puissance qui permet de fournir un courant alternatif stable et dépourvu de coupures ou de microcoupures, quoi qu'il se produise sur le réseau électrique (c.f \underline{\href{https://fr.wikipedia.org/wiki/Alimentation_sans_interruption}{Wikipédia}})}
}


\newglossaryentry{PIS}
{
	name=\underline{PIS},
	description={\textit{Passenger Information System}, Système d'Information des Passagers automatisé permettant de leur fournir à la fois des informations statiques comme des tables horaires ainsi que des informations dynamiques comme l'attente avant la prochaine rame ou encore les incidents survenus sur le réseau (c.f \underline{\href{https://en.wikipedia.org/wiki/Passenger_information_system}{Wikipédia}})}
}

\newglossaryentry{PAS}
{
	name=\underline{PAS},
	description={\textit{Public Address System}, désigne un système d'amplification et de distribution sonore électronique par le biais d'un microphone, amplificateur et de haut-parleurs, permettant à une personne de communiquer un message (pré-enregistré ou en direct) au grand public (c.f \underline{\href{https://en.wikipedia.org/wiki/Public_address_system}{Wikipédia}})}
}

\newglossaryentry{ACS-IDS}
{
	name=\underline{ACS-IDS},
	description={\textit{Access Control System-Intrusion Detection System}, Système de Contrôle des Accès ainsi que de Détection des Intrusions}
}

\newglossaryentry{FDS}
{
	name=\underline{FDS},
	description={\textit{Fire Detection System}, Système de Détection des Incendies}
}

\newglossaryentry{CCTV}
{
	name=\underline{CCTV},
	description={\textit{Closed-Circuit Television}, Système de Vidéosurveillance}
}

\newglossaryentry{SCADA}
{
	name=\underline{SCADA},
	description={\textit{Supervisory Control And Data Acquisition}, Système de Contrôle et d'Acquisition de Données, système de télégestion à grande échelle permettant de traiter en temps réel un grand nombre de télémesures, d'informations visuelles (caméras par exemple), d'alarmes, de contrôler à distance des installations techniques}
}

\newglossaryentry{MMS}
{
	name=\underline{MMS},
	description={\textit{Maintenance Management System}, Système de Management de la Maintenance}
}

\newglossaryentry{AFC}
{
	name=\underline{AFC},
	description={\textit{Automatic Fare Collection}, Système de Collection Automatique des Billets}
}

\newglossaryentry{WA}
{
	name=\underline{WA},
	description={\textit{Wifi Access}, Système permettant de proposer aux usager du LRT de bénéficier d'un accès à Internet via un un réseau WiFi}
}

\newglossaryentry{SnagR}
{
	name=\underline{SnagR},
	description={Solution en ligne permettant l'édition et la gestion des rapports de test ainsi que des différents défauts présents sur les systèmes du projet }
}


\newglossaryentry{Mezzoteam}
{
	name=\underline{Mezzoteam},
	description={Base de données regroupant l'ensemble des documents produits par les différents acteurs du projet}
}

\newglossaryentry{e-TOL}
{
	name=\underline{e-TOL},
	description={Plateforme de gestion de projet et de partage de fichiers interne à Thales, permettant par exemple de partager des fichiers entre l'équipe basée en France (offshore) et l'équipe basée à Lusail (inshore)}
}

\newglossaryentry{reporting}
{
	name=\underline{reporting},
	description={Ensemble des processus permettant d'aboutir à la présentation des activités du projet et de leurs avancements. Ici ce terme est utilisé pour désigner la gestion de la production des rapports de test}
}

\newglossaryentry{KPI}
{
	name=\underline{KPI},
	description={\textit{Key Performance Indicator}, Indicateur clé de performance, indicateur graphique et/ou numérique permettant d'évaluer l'avancement d'un projet, de communiquer, de diagnostiquer les points bloquants ou encore de s'assurer de la continuité du progrès}
}

\newglossaryentry{RAMS}
{
	name=\underline{RAMS},
	description={\textit{Reliability Availability Maintainability and Safety}, en français : FMDS (Fiabilité Maintenabilité Disponibilité et Sécurité), entité en charge de la sûreté de fonctionnement au sein d'un projet}
}

\newglossaryentry{IDE}
{
	name=\underline{IDE},
	description={\textit{Integrated Development Environment} en français :  Environnement de Développement Intégré (EDI) }
}

\newglossaryentry{LaTeX}
{
	name=\underline{LaTeX},
	description={LaTeX est un langage de description donnant à l'auteur les moyens d'obtenir des documents mis en page de façon professionnelle sans avoir à se soucier de leur forme. La priorité est donnée à l'essentiel : le contenu (c.f \underline{\href{https://openclassrooms.com/fr/courses/1617396-redigez-des-documents-de-qualite-avec-latex/1617565-quest-ce-que-latex}{OpenClassrooms}})}
}

\newglossaryentry{Git}
{
	name=\underline{Git},
	description={logiciel de gestion de versions décentralisé. C'est un logiciel libre créé par Linus Torvalds, auteur du noyau Linux, et distribué selon les termes de la licence publique générale GNU version 2.  (c.f \underline{\href{https://fr.wikipedia.org/wiki/Git}{Wikipédia}})}
}

\newglossaryentry{PCR}
{
	name=\underline{PCR},
	description={\textit{Product Change Request}, requête formulée par ingénieur, souvent par le biais d'un outil de gestion, afin de demander des changements dans le design d'un système}
}

\newglossaryentry{methodesagiles}
{
	name=\underline{méthodes agiles},
	description={Les méthodes agiles sont des groupes de pratiques de pilotage et de réalisation de projets. Elles ont pour origine le manifeste Agile, rédigé en 2001, qui consacre le terme agile pour référencer de multiples méthodes existantes. (c.f \underline{\href{https://fr.wikipedia.org/wiki/Méthode_agile}{Wikipédia}})}
}

\newglossaryentry{scraping}
{
	name=\underline{scraping},
	description={Le web scraping (parfois appelé harvesting) est une technique d'extraction du contenu de sites Web, via un script ou un programme, dans le but de le transformer pour permettre son utilisation dans un autre contexte. (c.f \underline{\href{https://www.luxury-concept.com/dev-blog/346-le-scaping-des-donnees-pourquoi-et-comment.html}{LuxuryConcept}})}
}

\newglossaryentry{TestCases}
{
	name=\underline{Test Cases},
	description={en français : scénarios de test, éléments composant les test, définis au moment du design afin de simuler plusieurs situations d'utilisation des systèmes de manière à prouver leur conformité vis à vis de toutes les exigences du cahier des charges}
	}

\newglossaryentry{SIT}
{
	name=\underline{SIT},
	description={\textit{Site Integration Test}, test d'intégration, ayant pour but de détécter les erreurs non détéctables pendant les test unitaires (c.f StAT). Ce type de test consiste à assembler différents composants afin de tester leur fonctionnement dans l'ensemble}
	}	
	
\newglossaryentry{E2E}
{
	name=\underline{E2E},
	description={\textit{End to End Test}, test bout en bout ayant pour but de vérifier si un système se comporte comme prévu du début à la fin. Le testeur doit se mettre dans le rôle d’un utilisateur et effectuer les tests comme s’il utilisait véritablement l’outil mis à sa disposition (c.f \underline{\href{https://blog.testingdigital.com/quest-test-de-bout-bout-end-to-end-1288}{TestingDigital}})}
	}

\newglossaryentry{SC}
{
	name=\underline{Safety Case},
	description={document indexant l'ensemble de exigences liées à la sécurité}
	}
	
\newglossaryentry{SIL}
{
	name=\underline{SIL},
	description={\textit{Safety Integrity Level}, niveau d'intégrité de sécurité, niveau relatif de réduction de risques inhérents à une fonction de sécurité, ou comme spécification d'une cible de réduction de risque. Plus simplement, c'est une mesure de la performance attendue pour une fonction de sécurité. (c.f \underline{\href{https://fr.wikipedia.org/wiki/Safety_Integrity_Level}{Wikipédia}})}
}

\newglossaryentry{API}
{
	name=\underline{API},
	description={\textit{Application Programming Interface }, Interface de Programmation Applicative, permettant à un logiciel d'offrir un accès facilité à certaines de ses fonctions, méthodes et classes}
}



\glsaddall

